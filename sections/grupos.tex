\documentclass[../template.tex]{subfiles}

\begin{document}

\section{Grupo}
\dfn{Grupo}{
Un grupo es un conjunto con una operación ($f(x,y)$) que satisface algunas condiciones.
\begin{itemize}
\item De un elemento del conjunto se puede llegar al otro con la operación.
\item Todo elemento del conjunto tiene un inverso.
\item Existe un elemento neutro.
\item La operación de dos elementos esta dentro del conjunto ($f(x,y) \in A \forall x,y \in A$)
\end{itemize}
}
\nt{Si la operación es conmutativa se le conoce como un Grupo Abeliano.}
\ex{$\mathbb{Z}$ y $+$}{
	$\mathbb{Z}$ y la suma ($f(x,y) = x+y$) son un grupo abeliano.
	\begin{itemize}
	\item Existe un neutro, 0. $f(x,0) = x$
	\item Se puede llegar a todos los números desde uno dado
		\begin{align}
			\intertext{Con $x$ y $z$ dados y $x$ no es el neutro}
			f(x,y) &= z\\
			x+y &= z\\
			y &= z-x
			\intertext{Entonces encontramos un número $y\neq z$ por el cual podemos llegar a cualquier $z$ desde una $x$}
		\end{align}
	\item Todo número tiene su inverso.
	\item La suma es conmutativa.
	\end{itemize}
}

\end{document}

